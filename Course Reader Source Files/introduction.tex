\thispagestyle{empty}
%----------------------------------------------------------------------
\chapter{Introduction}
\label{introduction.chap}
%----------------------------------------------------------------------

\section{Objectives}
%----------------------------------------------------------------------

This practical course reader aims at giving good advice to students realizing a project. Starting with the theoretical knowledge contained in the reader of Prof. Alexandre Bayen, students have to realize a project based on a distributed parameter system that they should:
\begin{enumerate}
	\item describe
	\item model
	\item explain
\end{enumerate}
The first part is a descriptive (i.e. qualitative) description of the system to be studied, its interest and the goal of the study. The second part is an elaboration, under simplifications, of a quantitative model (meaning here some kind of Partial Differential Equation and some control or optimization). The third part is an analysis of this model, either by solving the equations, by simulation or by a combination of both, using the tools learned during the course. It should lead to a better understanding of how the system works and how it can be optimized or controlled to fulfill the goals set in the description.

Therefore this practical course reader shows examples of modeling and various analysis methods that can support the creativity of the students realizing a project.
